\documentclass[12pt, a4paper]{report}
\usepackage{fullpage}
\usepackage{graphicx}
\usepackage{standalone}
\usepackage{float}
\usepackage{url}

\begin{document}

\begin{center}
\includegraphics[scale=0.6]{images/logo-uva.png}
\vspace{30pt}

\includegraphics[scale=0.2]{images/logo-sne_black-inv-flat}
\vspace{10pt}

\Large MSc. System and Network Engineering
\vspace{100pt}

\textbf{\huge SSN Project}
\vspace{10pt}

\textit{\Large "SSL session key extraction from memory on Android mobile devices"}
\vspace{80pt}

\large Stamatios Maritsas - Stamatios.Maritsas@os3.nl

\large Yadvir Singh - Yadvir.Singh@os3.nl

\large Kenneth van Rijsbergen - Kenneth.vanRijsbergen@os3.nl
\vspace{80pt}

\normalsize December, 2015
\end{center}

\abstract{asjhdjahfslkdffsd}

\tableofcontents

\chapter{Introduction}
\section{Your Section title Here}

\chapter{Related Work}
The article by Gursev Singh Kalra, titled ”Extracting RSAPrivate-
CrtKey and Certificates from an Android Process”, describes how to
dump X.509 certificates and construct a RSA private key (RSAPrivate-
CrtKey) from the Android application memory using Eclipse Memory
Analyzer Tool (MAT) and Java code. This paper gave us the indication that there are possibilities to extract the keys from a running
process.\cite{ref1}

\chapter{Approach}
\section{Heap dumping}
\section{Dynamic code instrumentation (Frida)}

\chapter{Experiments}
\section{Setup}
\subsection{Traffic capture}
\subsubsection{Proxy server}
\subsubsection{Wireshark}
\subsection{Desktop/Smartphone Setup}

\chapter{Results}

\chapter{Conclusion}

\chapter{Attack Limitations}

\chapter{Contribution} 

\bibliographystyle{plain}
\bibliography{bibliography}

\end{document}