

\documentclass[11pt]{article}
\usepackage{framed}
\usepackage{array}
\usepackage{eurosym}
%Gummi|065|=)
\title{\textbf{SSN project proposal}}
\author{Stamatis Maritsas\\
		Kenneth van Rijsbergen\\
		Yadvir Singh}
\date{}
\begin{document}

\maketitle


\abstract{Today RFID tags are used in many places where a user has to be authenticated. Use cases include granting access to rooms or making payments off the card holder's bank account. The University of Amsterdam (UvA) uses RFID technology in their student identification cards. These cards grant access to particular rooms and and are able to make payments. This research will investigate the security of these cards and the possible weaknesses.}
\section{Research question}


\begin{framed}
\noindent \textit{Is the UvA card duplicable and can information on the card be manipulated to get different user rights?}
\end{framed}
Our research will be include the following steps:

\begin{enumerate}
\item{Investigate what can be read from the cards and how data is structured.}
\item{Make a copy of UvA card and investigate the behaviour when using university facilities. This would
reveal what information is used by the facilities (UID of the card or some other information)}
\item{Finally, investigate the possibilities of manipulating data on the card (including a manipulated copy).}
\item{Investigate the security weaknesses of related facilities like door mechanisms and payment facilities.}
\end{enumerate}

\clearpage



\section{Related work}
Nowadays, we have more and more applications that are based on RFID implementation but the big question that arises is "How secure is this mechanism?". Although, it is safe to say that this technology has matured enough to be used in toll-gate payment systems, automobile security, casino management, animal tracking etc., it would not be wise to say that RFIDs are secure enough when it comes down to financial transactions or using ID documents embedded with RFID chips unless better safeguards are implemented to ensure foolproof security. \cite{grover2011survey}
\newline
\newline
The paper by Ortiz Jr. stated that there are a number of security issues that arise with the use of RFID. They can for example be used to send malicious data to a back end server. SRI International’s Neumann stated further that terrorist could copy cards that give access to RFID-based airport luggage-scanning systems. There is also a possibility that hackers "could record data from an RFID chip that is on an inexpensive product and upload the data to a chip that is on an expensive product, thereby getting the latter for a lower price" says Lukas Grunwald, a con-sultant with DN-Systems Enterprise Solution. \cite{ortiz2006secure}

\section{Planning}

\begin{center}
\begin{tabular}{ | c | m{11cm}|  } 
\hline
Week & Work \\ 
\hline
1 & Start by looking at RFID implantation and data structure. Look at the data dump of the UvA card. Possible reverse engineering?  \\ 
\hline
2 & Investigate the possibilities of copying cards, either complete a complete copy (byte by byte) or partial copies. Test the copied cards at door locks (find out which data is used for access rights). If we succeed in manipulating the door locks, we can repeat the test for other facilities like vending machines, printers etc.....  \\ 
\hline
3 & Investigate the possibilities of modifying data structures of the card and run the tests again (acquiring different access rights). \\ 
\hline
4 & Writing the report, making conclusions \& writing down test results, preparing the presentation. \\
\hline
\end{tabular}
\end{center}

\clearpage

\section{Hardware}
We intended to use a simple RFID reader/writing, a cheap solution would be to use an Arduino in combination with a simple RFID module (less than \euro{10}) in combination with blank RFID cards. 

\section{Ethical issues}
The first main issue that arises is the fact that if we can duplicate a card the privacy of the card holder is compromised and the basis of unique identification is undermined. This poses a risk of being exploited by hackers.

Secondly, if we are able to manipulate data, it would lead to possible attacks from malicious users that might want to get access or payment rights of other users.

\bibliographystyle{plain}
\bibliography{bibliography}

\end{document}
